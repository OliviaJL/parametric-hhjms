\section{Conclusions and Discussion}
\label{sec:discuss}

This report extends the joint model for longitudinal data of mixed types with truncated values and survival models by replacing the semi-parametric Cox proportional hazard model by two parametric survival models. It extends the semi-parametric joint modeling framework to parametric framework. The advantage of parametric framework is that the estimates can be more efficient if the distributional assumption of parametric models hold. The performance of parametric models is not guaranteed to be more accurate than semiparametric or nonparametric models. 

The simulation study shows that the parametric Weibull regression model is not necessarily to be more efficient than semi-parametric.

The joint model can be further extended by replacing the mixed effects models into other models, e.g., robust mixed effects models, generalized estimating equations (GEE models), and Bayesian mixed effects models. Other types of survival models are available as well, e.g., 

Compared to proportional hazards models, the accelerated failure time models are more straightforward as they directly measure the relationship between the log of time-to-event without introducing the hazards functions. The models are easier to interpret as well: the change of covariates directly explains the change of the expected log of the survival time. 


Survival models can be extended to frailty models by introducing time-dependent covariates and random effects. %compare using ph or AFT - page6