\section{Conclusions and Discussion}
\label{sec:discuss}

This report extends the joint model for longitudinal data of mixed types with truncated values and survival models by replacing the semi-parametric Cox proportional hazards model by two parametric survival models. It extends the semi-parametric joint modeling framework to parametric framework. An advantage of parametric framework is that the estimates can be more efficient if the distributional assumption of parametric models hold as the parametric models have less parameters and thus have much less asymptotic standard errors of the parameters. 

The approximation method, h-likelihood, is an important advantage of Yu's model as it yields relatively accurate results with a significantly improved computational efficiency compared to other approximate methods.
Another advantage of Yu's joint model is that it relaxes the unverifiable distributional assumption on truncated longitudinal data which is used in previous studies.

The joint model can be further extended by replacing the mixed effects models by other models, e.g., robust mixed effects models, generalized estimating equations (GEE models), and Bayesian mixed effects models. Other types of survival models are available as well, e.g., log-linear AFT models and frailty models if random effects are introduced. Compared to proportional hazards models, the accelerated failure time models are more straightforward as they directly measure the relationship between the log of time-to-event without introducing the hazards functions. The models are easier to interpret as well: the change of covariates directly explains the change of the expected log of the survival time. 