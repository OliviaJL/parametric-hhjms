\section{Methodology}
\label{sec:method}

We follow the notations in \cite{yu2018joint}. Suppose $Y$ is a continuous longitudinal variable where $Y_{ij}$ represents the value of $i$th participant at $j$th time point, where there are $n$ individuals, i.e., $i=1,\dots,n$ and $n_i$ time points for individual $i$, i.e., $j=1,\dots,n_i$. $Y$ is left-truncated by the LLOQ, i.e., $Y\geq d$, and denote $C$ as the corresponding truncation indicator, where $C=1$, if $Y\leq d$, and $C=0$, otherwise. Denote $Z$ as a discrete longitudinal variable, and $T_i^{\ast}$ as the observed survival data. The joint model proposed by Yu et al. \cite{yu2018joint} can be defined as follows.


\subsection{The Semi-parametric Joint Model}
\label{sec:2.1}

\vspace{2mm}
\noindent \textbf{\textit{Mixed Effects Models for Longitudinal Data}}
\vspace{2mm}

The longitudinal response $Y$ is continuous, and left-truncated due to LLOQ.
Capture the information of $Y$ in separate models:
\begin{equation}
    Y_{ij}|Y_{ij}\geq d = g({\Psi}_{ij}, {\Phi}_{ij}, \boldsymbol{\beta}, \boldsymbol{b}_{1i}) + \epsilon_{ij},
\end{equation}
where $g(\cdot)$ is a known linear or nonlinear function, ${\Psi}_{ij}\in \mathbb{R}^{p_1}$ and ${\Phi}_{ij}\in \mathbb{R}^{q_1}$ are vectors of covariates for participant $i$ at time $t_{ij}$, with $p_1$ time-independent variables and $q_1$ time-dependent variables respectively, $\boldsymbol{\beta}\in \mathbb{R}^{p_1}$ contains $p_1$ fixed parameters, $\boldsymbol{b}_{1i}\in \mathbb{R}^{q_1}$ contains $q_1$ random effects for participant $i$, with $i=1,...,n$, $j=1,...,n_i$. When $g({\Psi}_{ij}, {\Phi}_{ij}, \boldsymbol{\beta}, \boldsymbol{b}_{1i}) = {\Psi}_{ij}^T \boldsymbol{\beta} + {\Phi}_{ij}^T \boldsymbol{b}_{1i}$, it is a linear mixed effects model. The random effects for participant $i$ are independent and identically distributed as a multivariate normal distribution where $\boldsymbol{b}_{1i}\stackrel{iid}{\sim} N(\boldsymbol{0}, \boldsymbol{D}_1)$, and errors $\boldsymbol{\epsilon}_i\in \mathbb{R}^{n_i}$ follows $\boldsymbol{\epsilon}_{i}\stackrel{iid}{\sim} N(\boldsymbol{0}, {R}_i)$, ${R}_i=\sigma^2 {I}_i$.


$Z$ is discrete and modeled by a generalized linear mixed effects model:
\begin{equation}
    q(\mathbb{E}(Z_{ik})) = \boldsymbol{x}_{ik}^T \boldsymbol{\alpha} + \boldsymbol{\mu}_{ik}^T \boldsymbol{b}_{2i},
\end{equation}
where $\mathbb{E}(Z_{ik})$ represents the expectation of variable $Z_{ik}$ for participant $i$ at time $t_{ik}$, $q(\cdot)$ is a known link function, $\boldsymbol{x}_{ik}\in \mathbb{R}^{p_2}$ is a vector of $p_2$ time-independent variables, $ \boldsymbol{\mu}_{ik}\in \mathbb{R}^{q_2}$ is a vector of $q_2$ time-dependent variables for participant $i$ at time $t_{ik}$, $\boldsymbol{\alpha}\in \mathbb{R}^{p_2}$ contains $p_2$ fixed parameters, $\boldsymbol{\mu}_{ik}\in \mathbb{R}^{q_2}$ contains $q_2$ random effects for participant $i$, with $i=1,...,n$, time $k=1,...,m_i$.

Define the truncation indicator $C_{ij}$, for $Y_{ij}$ which is left-truncated by LLOQ $d$, i.e., $Y_{ij}\geq d$. 
$C_{ij} = I(Y_{ij}<d)$, and model it by
\begin{equation}
    \text{logit}(P(C_{ij} = 1)) = \boldsymbol{w}_{ij}^T \boldsymbol{\eta} + \boldsymbol{v}_{ij}^T \boldsymbol{b}_{3i},
\end{equation}
where $\boldsymbol{w}_{ij}\in \mathbb{R}^{p_3}$ is a vector of $p_3$ time-independent variables, $\boldsymbol{v}_{ij}\in \mathbb{R}^{q_3}$ is a vector of $q_3$ time-dependent variables for participant $i$ at time $t_{ij}$, $\boldsymbol{\eta}\in \mathbb{R}^{p_3}$ contains $p_3$ fixed parameters, $\boldsymbol{b}_{3i}\in \mathbb{R}^{q_3}$ contains $q_3$ random effects for participant $i$ and follows $\boldsymbol{b}_{3i}\stackrel{iid}{\sim} N(\boldsymbol{0}, \boldsymbol{D}_3)$, with $i=1,...,n$, $j=1,...,n_i$. 


Let the dimension of random effects $q=q_1+q_2+q_3$, and then the random effects are $\boldsymbol{b}_i = (\boldsymbol{b}_{1i}^T, \boldsymbol{b}_{2i}^T, \boldsymbol{b}_{3i}^T)^T \in \mathbb{R}^{q}$ for participant $i$.



\vspace{2mm}
\noindent \textbf{\textit{Survival Models for Survival Data}}
\vspace{2mm}

Lei $T_i^{\ast}$ be the time to the event of HIV infection, $\mathcal{C}_i$ be the right-censoring time, and then the observed time is defined by $S_i = min\{ T_i^{\ast}, \mathcal{C}_i\}$, and the event indicator is $\delta_i = I(T_i^{\ast} \leq \mathcal{C}_i)$. Assume the censoring is non-informative, the observed survival data can be measured by $\{(s_i, \delta_i),i=1,\dots,n \}$.

The Cox proportional hazards model builds the relationship between the hazards proportions and linear predictors. It can be defined by
\begin{equation}
    log\Big[\frac{h_i(t)}{h_0(t)}\Big] = \boldsymbol{x}_{si}^T \boldsymbol{\gamma}_0 + \boldsymbol{b}_i^T  \boldsymbol{\gamma}_1,
\end{equation}
where $h_i(t)$ is a hazard function of the time to event, $h_0(t)$ is an unspecified baseline hazard function, $\boldsymbol{x}_{si}\in \mathbb{R}^{p_4}$ is a vector of $p_4$ time-independent covariates for participant $i$, $\boldsymbol{b}_i$ contains the random effects in mixed effects models for participant $i$, $\boldsymbol{\gamma}_0\in \mathbb{R}^{p_4}$ and $\boldsymbol{\gamma}_1\in \mathbb{R}^{q_1+q_2+q_3}$ contains fixed effects.


\vspace{2mm}
\noindent \textbf{\textit{Statistical Inference of the Semi-parametric Modeling Framework}}
\vspace{2mm}


Let the fixed parameters be $\boldsymbol{\theta}_1 = (\boldsymbol{\beta}^T,\boldsymbol{\eta}^T,\boldsymbol{\alpha}^T, \boldsymbol{\gamma}_0^T,\boldsymbol{\gamma}_1^T)^T$, dispersion parameters be $\boldsymbol{\epsilon} = (\sigma, vec({\Sigma}))^T$, where $vec({\Sigma})$ represents the vectorized variance-covariance matrix of random effects. 
The joint likelihood function of the joint model can be written as
\begin{equation}
    \begin{split}
        L_h (\boldsymbol{\theta}_1,\boldsymbol{\epsilon}) = \prod_{i=1}^n \int\{ f(\boldsymbol{y}_{i}|\boldsymbol{c}_{i}=0,\boldsymbol{b}_{1i},\boldsymbol{\beta},\sigma) f(\boldsymbol{c}_{i}|\boldsymbol{b}_{3i},\boldsymbol{\eta}) f(\boldsymbol{z}_{i}|\boldsymbol{b}_{2i},\boldsymbol{\alpha}) f(s_i,\delta_i|h_0,\boldsymbol{\gamma_0},\boldsymbol{\gamma_1}) f(\boldsymbol{b}_i|\Sigma)\} d \boldsymbol{b}_i.
    \end{split}
\end{equation}



\subsection{The Proposed Parametric Joint Model}

The proposed framework replace the semi-parametric Cox proportional hazards model by parametric models, e.g.,  Weibull regression models and Log-logistic accelerated failure time models. The AFT models build the relationship between the log of time-to-event and covariates, which is more straightforward and easy to interpret compared to PH models. As the proportional hazards assumption is relatively strong in many cases, the AFT models have less restricted assumptions as well. A comprehensive discussion is provided in \cite{wang2006estimation}.

\subsubsection*{\textit{Parametric Survival Models}}

The Weibull regression models have two alternative representations, a proportional hazards (PH) form and an accelerated failure time (AFT) form. For a clear comparison between the parametric models, we use the AFT forms. The Weibull regression model or the log-logistic AFT model can then be defined as a log-linear model:
\begin{equation}
    log(T_i) = \boldsymbol{x}_{si}^T \boldsymbol{\alpha}_0 + \boldsymbol{b}_i^T  \boldsymbol{\alpha}_1 +\sigma_{\circ} \varepsilon_i,
\end{equation}
where $T_i$ is the survival time, $\sigma_{\circ}$ is a scale parameter, $\boldsymbol{x}_{si}\in \mathbb{R}^{p_4}$ is a vector of time-independent variables for individual $i$, $\boldsymbol{\alpha}_0$ and $\boldsymbol{\alpha}_1$ are fixed effects, and $\varepsilon_i$ is a random variable. The random variable $\varepsilon_i = \frac{log(T_i) - \boldsymbol{x}_{si}^T \boldsymbol{\alpha}_0 - \boldsymbol{b}_i^T  \boldsymbol{\alpha}_1}{\sigma_{\circ} }$ has a density $f_0(\varepsilon_i)$, and a corresponding survival $S_0(\varepsilon_i)$.
Following Lawless in 2003 \cite{lawless2011statistical}, the likelihood function for individual $i$ is given by
\begin{equation}
    \begin{split}
        L_{si} (\boldsymbol{\alpha}_0,\boldsymbol{\alpha}_1,\sigma_{\circ}) &= \Big[ \frac{1}{\sigma_{\circ}} f_0(\varepsilon_i) \Big]^{\delta_i} S_0(\varepsilon_i)^{1-\delta_i}, \\
        &=  \Big[ \frac{1}{\sigma_{\circ}} f_0 \big( \frac{log(T_i) - \boldsymbol{x}_{si}^T \boldsymbol{\alpha}_0 - \boldsymbol{b}_i^T  \boldsymbol{\alpha}_1}{\sigma_{\circ} } \big) \Big]^{\delta_i} 
        S_0 \big( \frac{log(T_i) - \boldsymbol{x}_{si}^T \boldsymbol{\alpha}_0 - \boldsymbol{b}_i^T  \boldsymbol{\alpha}_1}{\sigma_{\circ} } \big)^{1-\delta_i},
    \end{split}
\end{equation}
and the log-likelihood is 
\begin{equation}
    \begin{split}
         \ell_{si} (\boldsymbol{\alpha}_0,\boldsymbol{\alpha}_1,\sigma_{\circ}) &= -\delta_i log(\sigma_{\circ}) + \delta_i log f_0(\varepsilon_i) + (1-\delta_i) S_0(\varepsilon_i) \\ 
         &= -\delta_i log(\sigma_{\circ}) + \delta_i log f_0(\frac{log(T_i) - \boldsymbol{x}_{si}^T \boldsymbol{\alpha}_0 - \boldsymbol{b}_i^T  \boldsymbol{\alpha}_1}{\sigma_{\circ} }) + (1-\delta_i) S_0(\frac{log(T_i) - \boldsymbol{x}_{si}^T \boldsymbol{\alpha}_0 - \boldsymbol{b}_i^T  \boldsymbol{\alpha}_1}{\sigma_{\circ} }).
    \end{split}
    \label{eq:logSurv}
\end{equation}
The Weibull and Log-logistic AFT model assumes the survival data follows a Weibull and log-logistic distributions respectively, and thus the log of survival time follows extreme-value and logistic distributions respectively. The survival for Weibull AFT model is $S_0(\varepsilon)=exp(-e^{\varepsilon})$, and for log-logistic AFT model is $S_0(\varepsilon)=(1+e^{\varepsilon})^{-1}$, and $f_0(\varepsilon)$ is the corresponding density function.

\subsubsection*{\textit{Joint Likelihood with Weibull Regression Model and Log-logistic AFT Models}}

Let the fixed parameters be $\boldsymbol{\theta}_2 = (\boldsymbol{\beta}^T,\boldsymbol{\alpha}^T,\boldsymbol{\alpha}_0^T, \boldsymbol{\alpha}_1^T, \boldsymbol{\eta}^T, \sigma_{\circ})^T$, and the vector of dispersion parameters is $\boldsymbol{\epsilon}=(\sigma, vec({\Sigma}))^T$ as define above in Sec \ref{sec:2.1}. 
The likelihood function of the joint model is
\begin{equation}
    \begin{split}
        L_{h}(\boldsymbol{\theta}_2,\boldsymbol{\epsilon}) = \prod_{i=1}^n \int\{ f(\boldsymbol{y}_{i}|\boldsymbol{c}_{i}=0,\boldsymbol{b}_{1i},\boldsymbol{\beta},\sigma) f(\boldsymbol{c}_{i}|\boldsymbol{b}_{3i},\boldsymbol{\eta}) f(\boldsymbol{z}_{i}|\boldsymbol{b}_{2i},\boldsymbol{\alpha}) f(s_i,\delta_i|\boldsymbol{\alpha}_0,\boldsymbol{\alpha}_1,\sigma_{\circ}) f(\boldsymbol{b}_i|\Sigma)\} d \boldsymbol{b}_i,
    \end{split}
\end{equation}



\subsection{Approximation for Joint Likelihood Inference}

The common way to solve the joint likelihood is the Monte Carlo EM (ECM) algorithms; however, when the vector of parameters has a very high dimension, it converges to the approximate estimates very slowly, and can fail to converge in many cases. A more feasible way is the numerical integration methods such as Gauss-Hermite quadrature method, but it can be computationally consuming in high dimensions. An alternative is the h-likelihood method, which is a numerical integration method as well, but can be much more computationally efficient with reasonable accuracy \cite{ha2003joint,molas2013joint}. The h-likelihood method thus is used to approximate the joint likelihood. By using the h-likelihood method, the integration can be avoided by treating the random effects as parameters in the log-likelihood function. The log h-likelihood function of individual $i$ can be written as the sum of logarithmic terms which contain the term of random effects:
\begin{equation}
\begin{split}
    \ell_{hi} (\boldsymbol{\theta}_2, \boldsymbol{\epsilon}, \boldsymbol{b}_{i}) =&\ log (f(\boldsymbol{y}_{i}|\boldsymbol{c}_{i}=0,\boldsymbol{b}_{1i},\boldsymbol{\beta},\sigma) )
    +\sum_{j=1}^{n_i} \Big[ (1-c_i) logP(c_{ij}=0|\boldsymbol{b}_{3i},\boldsymbol{\eta}) + c_i logP(c_{ij}=1|\boldsymbol{b}_{3i},\boldsymbol{\eta})\Big] +  \\
    &\sum_{j=1}^{n_i} \Big[ (1-z_i)P(z_{ij}=0|\boldsymbol{b}_{2i},\boldsymbol{\alpha}) + z_iP(z_{ij}=1|\boldsymbol{b}_{2i},\boldsymbol{\alpha}) \Big] 
    + log( f(\boldsymbol{b}_i))  + \ell_{si} (\boldsymbol{\alpha}_0,\boldsymbol{\alpha}_1,\sigma_{\circ}),
\end{split}
\end{equation}
where $\ell_{si} (\boldsymbol{\alpha}_0,\boldsymbol{\alpha}_1,\sigma_{\circ})$ is the log-likelihood of AFT models in Eq.(\ref{eq:logSurv}).


The algorithm solves the likelihood function by updating the random effects, fixed effects and dispersion parameters sequentially in each iteration. The choice of initial values of parameters is important as the joint likelihood can be non-identifiable and the optimal estimates are not identical, and the model can hardly be strictly convex and it is easy to yield local optima and fail to converge to the global optimization. It is reasonable to find the initial values by modeling the multiple processes in separate models, and the estimates can then be used as a good start of joint inference. Since the standard errors of parameter estimates are underestimated using the h-likelihood method, an adaptive Gauss-Hermite (aGH) method is proposed to re-estimate the standard errors.
 