\documentclass[10pt]{article}
% header.tex

\usepackage[left=1in,right=1in,top=1in,footskip=25pt]{geometry} 
\usepackage{amsmath,amsthm,amssymb,amsfonts}
\usepackage{mathtools}
\usepackage{enumitem}
\usepackage{dsfont}
\usepackage{centernot}
\usepackage[usenames,dvipsnames]{xcolor}
\usepackage{appendix}

\usepackage{setspace}
\doublespacing

%slashbox in tables
\usepackage{makecell}
% set up graphics
\usepackage{graphicx}
\DeclareGraphicsExtensions{.pdf,.png,.jpg}
\graphicspath{ {fig/} }

\usepackage[sorting=nyt,backend=biber,bibstyle=alphabetic,citestyle=alphabetic,giveninits=true]{biblatex}

\usepackage{fancyhdr}
\pagestyle{fancy}
\fancyhead[L]{}
\setlength{\headheight}{40pt}
\usepackage{multirow}

% set up hyperlinks:
\definecolor{RefColor}{rgb}{0,0,.65}

\usepackage[colorlinks,linkcolor=RefColor,citecolor=RefColor,urlcolor=RefColor]{hyperref}
\usepackage[capitalize]{cleveref}
\crefname{appsec}{Appendix}{Appendices} % you can tell cleveref what to call things

% Pseudo-codes:
\usepackage{algorithm,algpseudocode}

\title{An Extension of a Joint Model for Longitudinal Data of Mixed Types and Survival Data:\\
\textit{from Semi-parametric to Parametric Framework}
}
\author{Jiaping(Olivia) Liu}

\addbibresource{hhjms.bib} 


\begin{document}
\maketitle

\begin{abstract}
    This report provides an extension of the joint model for mixed and truncated longitudinal data and survival data proposed by Yu et al. in 2018 \cite{yu2018joint} by replacing the semi-parametric survival model by a couple of parametric survival models. They have similar performance on the real dataset. In the simulation study, proportional hazards models yield similar results, and performs poorly in the estimation of same parameters. The issue might be due to the invalid assumption of proportional hazards.
    
    \noindent KEYWORDS: Weibull regression model, Accelerated failure time model, h-Likelihood, parametric model, computational efficiency
\end{abstract}

\section{Introduction}
\label{sec:intro}

%big picture: 
In longitudinal studies it is common that there are more than one response of interest. The responses can have multiple types. They can be longitudinal or survival data, continuous or discrete, and they usually relate to each other. It requires, therefore, modeling frameworks to measure the mixed types of responses and their relationships simultaneously. Computation of such models can be complicated as it usually contains an integration of high-dimensional parameters, which is intractable to solve.
Moreover, there are problems of incomplete data: missing values, outliers, measurement errors and censoring, which require special treatments in the models, and makes it more complex to solve the joint models.


% motivation case & special data type -> need special treatment/model
The study by Yu et al. \cite{yu2018joint} is motivated by the experiment of HIV immunization in which two longitudinal processes and a survival process are of the main interest. There is a truncated, continuous longitudinal response, a binary longitudinal process, and a survival process, all of which are correlated to each other.
% the HIV case: %(refer to the paper for more details of the case)
%there are two longitudinal responses that are of the main interests: a truncated continuous response NAb and a binary response; and a survival response
%The responses are related to each other.
As there is no one response that is more important than others, and they all have different types, the responses can be modeled separately in different models, and then correlate to each other by shared parameters, e.g., random effects.
%Therefore, this case requires a joint model that can capture both the data of mixed types and the relationship among responses simultaneously.


\subsection{Joint Models, Statistical Inference and Common Problems}

Model specification is important as the types of responses are complex and correctly measure the data is important. Statistical inference based on the joint model can be complicated as the joint likelihood inference has a high-dimensional integration of parameters which is intractable. It is common that the longitudinal and survival data are incomplete. Longitudinal data involve missing values, outliers, measurement errors, and they can be censored, and survival data can be censored. the incomplete data problems require special treatment in the joint models.

% mixed types of responses & correlation among them
\subsubsection*{\textit{Model Specification and Literature Review}}

In the cases that there are multiple longitudinal and survival processes of interest, researchers, first of all, need to identify the primary and secondary processes. The correct specification of models for primary processes is critical, and the secondary models can be simplified by setting the parameters as nuisance parameters. High-dimensional parameters can easily cause the problem of model non-identifiability in statistical inference, in which difference sets of parameters can lead to same performance. A more parsimonious model, therefore, can potentially avoid this issue and lead to more efficient statistical inference. 
If both longitudinal and survival processes are of the main interest, the models for longitudinal and survival data, for example, can be linked by shared parameters. The association parameters may be reduced.
%they may share the same random effects since these random effects characterize the individual-specific longitudinal processes.

%If the longitudinal data and survival data are both of interest in the case, they need to be modeled jointly to study their relationship and avoid possible bias. 
%
%measure responses separately in different models, and connect them by sharing parameters or variables.
% longitudinal
For two or more longitudinal processes, they may be treated as responses in a single multivariate model or in separate models, or responses and time-dependent covariates in a single model. 
%In the former case we are interested in their association over time, and in the latter case we are 
Models for longitudinal responses must address both the relationship between the response and covariates and the correlation in the repeated measurements. They must measure two sources of variations: within-individual and between individual variations. There are three common models to measure longitudinal data: mixed effects models, marginal GEE models, and a transitional modelling approach. An advantage of mixed effects models is they allow individual-specific inference. A restriction is mixed effects models have distributional assumptions which may not hold in some cases. Random effects in the mixed effects models represent the influence of each individual on the repeated observations that is not captured by the observed covariates. They accommodate the heterogeneity in the data, which may arise from subject or clustering effects or from spatial correlation. The magnitude of the random effects measures the variability across individuals or measures the between-individual variations.
%Mixed effects models are called subject-specific models, which is contrasted with the population averaged models or marginal models.
Semi-parametric or nonparametric mixed effects models relax the distributional assumptions.

% survival
Survival regression measures the dependence of event times on covariates. There are two main kinds of survival models: proportional hazards (PH) models and accelerated failure time (AFT) models. PH models have an assumption of constant hazard ratio over time which is a proportion of the hazard function of survival data and the baseline hazard. A commonly used PH model is Cox PH model, which is semi-parametric and do not have distributional assumption. Another common model is Weibull regression model, which can be treated either as a PH or a AFT model. Weibull regression models are assumed the survival data follow a Weibull distribution. Whereas, AFT models are parametric models. AFT models are more interpretable than PH models as AFT models can be interpreted as the speed of disease progression. It relaxes the proportional hazards assumption. 
Parametric survival models are more efficient if distributional assumptions hold, whereas the distributional assumptions can be restricted sometimes. %AFT models, compared to PH models, offer better interpretations on survival data.


If the longitudinal and survival data are associated, they need to modeled jointly to avoid biased estimates. They can be linked by shared parameters or shared variables. Linked mixed effects models and frailty models, for example, can lead to shared random effects models. If the models are governed by same underlying process, shared variables models are suitable.
% correlation: dependent parameters or dependent covariates
% random effects & correlation among responses via random effects
Lawrence et al in 2015 discusses the existing joint models comprehensively \cite{lawrence2015joint}. For truncated longitudinal processes, Bernhardt et al \cite{bernhardt2014flexible} proposed a joint model which treats the truncated data as a covariates in AFT survival models \cite{bernhardt2014flexible}.

% joint model for the different types of related responses & literature review
% literature review of the models that solve similar problems, but not quite suitable for the case (advantages and drawbacks)




\subsubsection*{\textit{Joint Inference using Exact and Approximate Methods}}

%joint inference is required.

A simple way to do statistical inference of joint model is a two-step method, which estimates the shared variables or parameters in one model, and then estimates the parameters in the other model separately using the estimated shared variables or parameters. The method is easy to implement as software is  available. It, however, can lead to biased estimates, especially when the longitudinal data and survival data are strongly correlated. It underestimates the uncertainty as the uncertainty of estimates in the first step is not incorporated in the second step. A third step, therefore, is usually needed to correct the bias and incorporate the estimation uncertainty in the first step, but it can be difficult for some complex problems.  
%it is also called regression calibration method in measurement error literature.
%
A more efficient joint inference method is preferred.


A key difference between LME and GLMM/NLME/Frailty is that the random effects are linear in the LME models, but nonlinear in the others. This leads to major computational challenges in likelihood estimation for models nonlinear in the random effects, since the likelihoods involve..
the difficulties: GLME models are used; therefore, cannot find a closed-form or analytic expressions of parameter estimates
%
Joint likelihood inference provides valid and reliable inference, and the resulting maximum likelihood estimates (MLEs) are asymptotically efficient and asymptotically normal under the usual regularity conditions. 
%advantages
MLEs have nice asymptotic properties that they are consistent, asymptotically normal and asymptotically efficient under some conditions
the likelihood inference of mixed effects models are conceptually straight-forward and MLEs have very attractive asymptotic properties
%
It requires the model or parameter identifiability, but there are many unknown parameters and two sets of parameters are likely to yield same likelihood. 
Joint likelihood for longitudinal and survival models can be highly complicated as the integration with respect to high-dimensional, unobservable random effects can be intractable. %, censoring and the semiparametric or nonlinear structures of the models
%it can be intractable except for LME models.
% Computation of likelihood function of such models is challenging as there are high-dimensional parameters and the integration of high-dimensional random effects is intractable.
%
%a drawback is the distributional assumption
%drawback: the likelihood method requires strong assumptions such as linearity
The likelihood inference may be sensitive to departures from the assumed distributions and sensitive to outliers.
%GEEs are less efficient than the likelihood estimates if the distributional assumptions hold
%quasi-likelihood method are more robust, and closely related to GEEs in which only the first two moments are needed to specify
%
%a drawback is the distributional assumption
%drawback: the likelihood method requires strong assumptions such as linearity
The likelihood inference may be sensitive to departures from the assumed distributions and sensitive to outliers.
%GEEs are less efficient than the likelihood estimates if the distributional assumptions hold
%quasi-likelihood method are more robust, and closely related to GEEs in which only the first two moments are needed to specify
GEE estimates are consistent as long as the mean structure is correctly specified, even the covariance structure is misspecified; asymptotically normal; not fully efficient


% measurement: exact and approximate solutions
% EM algorithm, Laplacian - h-likelihood
There are three main types of methods to estimate the joint likelihood:  ``exact'' methods, EM algorithm and approximate methods.
The ``exact'' methods, e.g., the Gauss-Hermite quadrature method or Monte Carlo methods, use submission to measure the integration with arbitrary accuracy. Its computation is intensive and even infeasible in high-dimensional cases, so it works well in relatively low-dimensional cases when random effects are normal.
The Monte Carlo EM (ECM) algorithm can be used on any dimensions of parameters and any distributions. It treats the random effects as the additional ``missing data'', so that the random effects can be estimated as parameters and do not need to be integrated out. A downside is the ECM algorithm converges very slowly and it is easy to fail to converge.
%It uses an E-step and an M-step to ..until convergence, but it still suffers the problem of slow convergence and intensive computation.
The approximate methods, e.g., Laplace or Taylor approximations, avoid solving the integration and can be much more computationally efficient.
%There are Monte Carlo EM algorithm for exact likelihood inference that can be computationally intensive, and approximate methods that are much more efficient in computation.
%the approximate method consists of a first-order Laplace approximation
%
Given the non-identifiability of the joint likelihood, good starting values are important as it can easily converge to a local optimization and fail to converge to the global optimization. 
%Other problems: model selection, covariate selection, parameter selection, etc.
%model selection: use parsimonious models to avoid potential problem of collinearity, and improve the precision of the main parameter estimates  %(why improve precision?)


% incomplete data problems
\subsubsection*{\textit{Incomplete Data Problems in Joint Modeling}}

% in separate models
Given the unique characteristics of longitudinal or survival study, there are several common incomplete data problems \cite{wu2009mixed}, such as missing data, measurement errors, outliers and censoring, which may potentially lead to biased data analysis results. Variables are measured over a period of time, and thus some subjects may fail to provide information at some time points in the middle of the study, which leads to a non-monotone missing pattern, or drop out the study before the last measurement, which results in a monotone missing pattern. Censoring exists when very large or small values are not measurable in practice. Outliers may exist and be influential to the results of some statistical methods, and thus need to be identified to reduce the bias in the analysis. There are two types of outliers in a longitudinal study: outliers of repeated measurements within individual subjects, and outliers in a variable at fixed time points. 
%Other data problems may exist in practice, but the four incomplete data problems are the most common ones, and thus are the main focus in the report. 
%
Survival data are often censored as the event of interest may not be observed for some subjects throughout the study period, skewed as they are usually not symmetric, and often have unequal follow-up times.



% missing data
In separate longitudinal models, missing data can be easily identified, and there are multiple methods to eliminate the effect of missing values, e.g., complete removal and the last value carried forward (LVCF) method.
They are proper to use when the reason for missing is ignorable, i.e., data are missing at random (MAR) or completely at random (MCAR). However, they can lead to biased analysis results if the reason is nonignorable, i.e. data are missing not at random (MNAR). In this case, robust methods, such as multiple imputation (MI) method and expectation-maximization (EM) algorithm, are more appropriate to use as they can measure more uncertainty of missing values and lead to less biased results. %The analysis of missing data modeling is related to several aspects: the variable is the response or a covariate; it is time-dependent or time-independent of it is a covariate; if is categorical or quantitative; the reason for missing is ignorable or not. The selection of missing data methods also depends on the statistical modeling approaches. 
A practical method, called a sensitivity analysis, can be used to compare the results of simple and robust methods for missing data in modelling, and select the preferred method to address the missing data. 
% outliers
Outliers may be identified through plots sometimes, but it is inefficient in many longitudinal studies, e.g., when the number of subjects is very large. In this case, a heavy tail distribution, e.g., t-distribution, can be used to accommodate outliers. 
However, it is difficult to identify measurement errors of the observed data and censored data. Values on the boundary of the input space may be censored data, but may also be true observations. The impact of them may be eliminated by a sensitivity analysis comparing the results of a standard model and a robust models.


% in joint modeling:
Missing data and measurement errors can occur in longitudinal or survival data, and in responses or covariates.
An additional binary, longitudinal process can be included if there are nonignorably incomplete data in responses or time-independent covariates. Whereas if incomplete data occur in time-dependent covariates, we need additional processes to model the incomplete mechanisms of covariates. Both of the methods make the joint model more complicated. This can lead to two major problems. The computation can be more complicated. Therefore, approximate methods which are more computationally efficient are highly valuable for joint models. Another major issue is the parameter identifiability that different sets of parameters can lead to the same joint likelihood. In summary, the joint model needs to be as simple as possible to avoid the problems of non-identifiability in computation, and improve the precision of the estimates of the important parameters.
%
% Data type and potential measurement problems, variable relationship -> the kind of model needed
% what information should be measured in the data: data types, relationships, etc.
%Mixed effects models are straightforward to incorporate missing data or measurement errors in likelihood inference for mixed effects models, even when the missing data mechanism is non-ignorable.




% the proposed semi-parametric modeling framework HHJMs
\subsection{The Joint Model for Longitudinal Data of Mixed Types and Survival Data}


% model specification: 
%how the HHJMs deal with the mixed types of responses (advantages and drawbacks)
Yu et al. in 2018 \cite{yu2018joint} proposed a joint model for multiple longitudinal and survival processes, called HHJMs. HHJMs measure multiple longitudinal process including a truncated variable, as well as a survival process, by h-likelihood, which solves the high-dimensional integration approximately with high computational efficiency. It is motivated by the HIV vaccine data where the immunization biomarkers are truncated to be larger than a specific value. The proposed model measures the truncated longitudinal variable by a couple of mixed effects models, one of which measuring the longitudinal trend and the other one measuring the truncation indicator. It is similar to the treatment on nonignorably missing data that we use a generalized linear mixed effects model to measure the binary indicator. This method is superior than previous methods as it avoids unverifiable assumption on truncated values.

% h-likelihood
As is discussed in previous subsections, the computation and identifiability of the joint model are two major problems. HHJMs use an approximate method, called h-likelihood, to compute the joint likelihood which is much more computationally efficient than the exact methods, e.g., Gauss-Hermite method, or EM algorithm. The h-likelihood method treats the random effects as parameters. During the iteration, fixed effects, random effects and dispersion parameters are updated sequentially given the updated estimates from last iteration. The starting values of the parameters are critical, again, as the model can be non-identifiable, so that different sets of estimates can lead to same joint likelihood; furthermore, it is possible the parameters converge to sub-optimal solutions, or even stuck at dents that far from the optimal solutions and fail to converge after a large number of iterations.
%Instead of integrating out the random effects as in linear mixed effects models, it measures the sum of random effects.


% other problems: incomplete data, common solutions and how the HHJMs deal with them
%As in the HIV immunization case there are two types of longitudinal data, they cannot be modeled in a single multivariate model, and should be measured in separate models.
HHJMs assume the missing data are missing at random, and do not include a missing data mechanism into the joint model. 
A reason is that the HIV vaccine data are collected from limited subjects, so that the model should not be too complicated to overfit the data. More importantly, the joint model should be parsimonious for the computational efficiency, and the missing data mechanism can include several parameters into the model and load more burden on the computation.





% parametric HHJMs
\subsection{The Proposed Parametric Joint Model}
% the extension to eliminate the limitation (parametric HHJMs): similarity and differences to HHJMs, advantages and limitations
% models and approximation methods

HHJMs measure the survival data by a Cox proportional hazards model. It is a semi-parametric model that do not have distributional assumption on survival data, but have a proportional hazards assumption that the hazards of survival data over the unspecified baseline hazards have a constant proportion which equals the linear predictor of the survival model. An advantage of the Cox model is it does not have a restriction of distributions on data, but it can be less efficient if the distributional assumptions hold compared to parametric survival models.

It worth comparing the Cox model to parametric models, e.g., Weibull regression models and accelerated failure time (AFT) models. The Weibull regression model assumes the survival data follows a Weibull distribution and the baseline hazards  function is specified as a density of baseline Weibull distribution. It has the proportional hazards assumption as well. Parametric accelerated failure time models have distributional assumptions on survival data, and relaxes the proportional hazards assumption which may not be valid sometimes. This report focuses on the Weibull regression model and a parametric AFT model, Log-logistic AFT model, compares the performance of the three joint models using different survival models on real data. Simulation results are provided on the joint models using Cox and using Weibull models as limited time and computational resource.


\subsection{Overview of the Report}

This report explored the performance of the joint model of longitudinal data of mixed types and survival data, HHJMs, comparing to their parametric extensions. Section \ref{sec:intro} introduces the existing techniques for jointly modeling the longitudinal and survival processes, indicating the major problems relating to model identifiability, computation, and incomplete data in existing studies, discussing briefly the advantages and drawbacks of HHJMs and pointing out the possible extensions of HHJMs that will be discussed in the following of the report. Section \ref{sec:method} defined the HHJMs and the proposed parametric framework. Section \ref{sec:applc} use the HHJMs and their parametric extensions on a real dataset and compare their performance. Section \ref{sec:simul} compares the HHJMs and the extension to Weibull regression models in synthetic data. Section \ref{sec:discuss}, in the end, summarises the advantages and drawbacks of HHJMs and their parametric extensions, and discusses several further studies.

%?why propose a new approach to estimate the standard errors of the  h-likelihood based parameter estimates by using an adaptive Gauss-Hermite method??



% Lang's book
%\cite{wu2009mixed}

%longitudinal studies:
%multiple measurements of a variable on the same individuals are correlated
%study changes of variables over time
%the correlation among data should be incorporated in the analysis in order to avoid potential bias and loss of efficiency.
%- allow change over time, and allow unbalanced data
%- info from different individuals in statistical inference

%missing data and dropouts are common,
%which may not be the case in clustered data or repeated measurement data


%FOUR main problems: - missing data and dropouts;- measurement errors;- censoring;- outliers
%may lead to severely biased or misleading results
%solve the problems to have a reliable result.

%longitudinal study: 
%- unbalanced numbers of measurements and measurement times across individuals
%- within-individual repeated measurements are correlated
%- substantial variation in both within-individual and between-individual measurement
%- incomplete data: FOUR main problems

%in a regression model, covariates are used to partially explain the systematic variation in the response, the remaining unexplained variation in the response is treated as random and is often assumed to follow a probability distribution

%unknown parameters be estimated by least squares method or maximum likelihood method

%model checking or model diagnostics should be performed to check the reasonability of the model and the assumptions
%informally based on residual plots and other graphical techniques

%variable transformation may be used to improve model fitting

%outliers and influential observations should be checked since they may greatly affect the resulting estimates and may lead to misleading inference


\section{Methodology}
\label{sec:method}

We follow the notations in \cite{yu2018joint}. Suppose $Y$ is a continuous longitudinal variable where $Y_{ij}$ represents the value of $i$th participant at $j$th time point, where there are $n$ individuals, i.e., $i=1,\dots,n$ and $n_i$ time points for individual $i$, i.e., $j=1,\dots,n_i$. $Y$ is left-truncated by the LLOQ, i.e., $Y\geq d$, and denote $C$ as the corresponding truncation indicator, where $C=1$, if $Y\leq d$, and $C=0$, otherwise. Denote $Z$ as a discrete longitudinal variable, and $T_i^{\ast}$ as the observed survival data. The joint model proposed by Yu et al. \cite{yu2018joint} can be defined as follows.


\subsection{The Semi-parametric Joint Model}
\label{sec:2.1}

\vspace{2mm}
\noindent \textbf{\textit{Mixed Effects Models for Longitudinal Data}}
\vspace{2mm}

The longitudinal response $Y$ is continuous, and left-truncated due to LLOQ.
Capture the information of $Y$ in separate models:
\begin{equation}
    Y_{ij}|Y_{ij}\geq d = g({\Psi}_{ij}, {\Phi}_{ij}, \boldsymbol{\beta}, \boldsymbol{b}_{1i}) + \epsilon_{ij},
\end{equation}
where $g(\cdot)$ is a known linear or nonlinear function, ${\Psi}_{ij}\in \mathbb{R}^{p_1}$ and ${\Phi}_{ij}\in \mathbb{R}^{q_1}$ are vectors of covariates for participant $i$ at time $t_{ij}$, with $p_1$ time-independent variables and $q_1$ time-dependent variables respectively, $\boldsymbol{\beta}\in \mathbb{R}^{p_1}$ contains $p_1$ fixed parameters, $\boldsymbol{b}_{1i}\in \mathbb{R}^{q_1}$ contains $q_1$ random effects for participant $i$, with $i=1,...,n$, $j=1,...,n_i$. When $g({\Psi}_{ij}, {\Phi}_{ij}, \boldsymbol{\beta}, \boldsymbol{b}_{1i}) = {\Psi}_{ij}^T \boldsymbol{\beta} + {\Phi}_{ij}^T \boldsymbol{b}_{1i}$, it is a linear mixed effects model. The random effects for participant $i$ are independent and identically distributed as a multivariate normal distribution where $\boldsymbol{b}_{1i}\stackrel{iid}{\sim} N(\boldsymbol{0}, \boldsymbol{D}_1)$, and errors $\boldsymbol{\epsilon}_i\in \mathbb{R}^{n_i}$ follows $\boldsymbol{\epsilon}_{i}\stackrel{iid}{\sim} N(\boldsymbol{0}, {R}_i)$, ${R}_i=\sigma^2 {I}_i$.


$Z$ is discrete and modeled by a generalized linear mixed effects model:
\begin{equation}
    q(\mathbb{E}(Z_{ik})) = \boldsymbol{x}_{ik}^T \boldsymbol{\alpha} + \boldsymbol{\mu}_{ik}^T \boldsymbol{b}_{2i},
\end{equation}
where $\mathbb{E}(Z_{ik})$ represents the expectation of variable $Z_{ik}$ for participant $i$ at time $t_{ik}$, $q(\cdot)$ is a known link function, $\boldsymbol{x}_{ik}\in \mathbb{R}^{p_2}$ is a vector of $p_2$ time-independent variables, $ \boldsymbol{\mu}_{ik}\in \mathbb{R}^{q_2}$ is a vector of $q_2$ time-dependent variables for participant $i$ at time $t_{ik}$, $\boldsymbol{\alpha}\in \mathbb{R}^{p_2}$ contains $p_2$ fixed parameters, $\boldsymbol{\mu}_{ik}\in \mathbb{R}^{q_2}$ contains $q_2$ random effects for participant $i$, with $i=1,...,n$, time $k=1,...,m_i$.

Define the truncation indicator $C_{ij}$, for $Y_{ij}$ which is left-truncated by LLOQ $d$, i.e., $Y_{ij}\geq d$. 
$C_{ij} = I(Y_{ij}<d)$, and model it by
\begin{equation}
    \text{logit}(P(C_{ij} = 1)) = \boldsymbol{w}_{ij}^T \boldsymbol{\eta} + \boldsymbol{v}_{ij}^T \boldsymbol{b}_{3i},
\end{equation}
where $\boldsymbol{w}_{ij}\in \mathbb{R}^{p_3}$ is a vector of $p_3$ time-independent variables, $\boldsymbol{v}_{ij}\in \mathbb{R}^{q_3}$ is a vector of $q_3$ time-dependent variables for participant $i$ at time $t_{ij}$, $\boldsymbol{\eta}\in \mathbb{R}^{p_3}$ contains $p_3$ fixed parameters, $\boldsymbol{b}_{3i}\in \mathbb{R}^{q_3}$ contains $q_3$ random effects for participant $i$ and follows $\boldsymbol{b}_{3i}\stackrel{iid}{\sim} N(\boldsymbol{0}, \boldsymbol{D}_3)$, with $i=1,...,n$, $j=1,...,n_i$. 


Let the dimension of random effects $q=q_1+q_2+q_3$, and then the random effects are $\boldsymbol{b}_i = (\boldsymbol{b}_{1i}^T, \boldsymbol{b}_{2i}^T, \boldsymbol{b}_{3i}^T)^T \in \mathbb{R}^{q}$ for participant $i$.



\vspace{2mm}
\noindent \textbf{\textit{Survival Models for Survival Data}}
\vspace{2mm}

Lei $T_i^{\ast}$ be the time to the event of HIV infection, $\mathcal{C}_i$ be the right-censoring time, and then the observed time is defined by $S_i = min\{ T_i^{\ast}, \mathcal{C}_i\}$, and the event indicator is $\delta_i = I(T_i^{\ast} \leq \mathcal{C}_i)$. Assume the censoring is non-informative, the observed survival data can be measured by $\{(s_i, \delta_i),i=1,\dots,n \}$.

The Cox proportional hazards model builds the relationship between the hazards proportions and linear predictors. It can be defined by
\begin{equation}
    log\Big[\frac{h_i(t)}{h_0(t)}\Big] = \boldsymbol{x}_{si}^T \boldsymbol{\gamma}_0 + \boldsymbol{b}_i^T  \boldsymbol{\gamma}_1,
\end{equation}
where $h_i(t)$ is a hazard function of the time to event, $h_0(t)$ is an unspecified baseline hazard function, $\boldsymbol{x}_{si}\in \mathbb{R}^{p_4}$ is a vector of $p_4$ time-independent covariates for participant $i$, $\boldsymbol{b}_i$ contains the random effects in mixed effects models for participant $i$, $\boldsymbol{\gamma}_0\in \mathbb{R}^{p_4}$ and $\boldsymbol{\gamma}_1\in \mathbb{R}^{q_1+q_2+q_3}$ contains fixed effects.


\vspace{2mm}
\noindent \textbf{\textit{Statistical Inference of the Semi-parametric Modeling Framework}}
\vspace{2mm}


Let the fixed parameters be $\boldsymbol{\theta}_1 = (\boldsymbol{\beta}^T,\boldsymbol{\eta}^T,\boldsymbol{\alpha}^T, \boldsymbol{\gamma}_0^T,\boldsymbol{\gamma}_1^T)^T$, dispersion parameters be $\boldsymbol{\epsilon} = (\sigma, vec({\Sigma}))^T$, where $vec({\Sigma})$ represents the vectorized variance-covariance matrix of random effects. 
The joint likelihood function of the joint model can be written as
\begin{equation}
    \begin{split}
        L_h (\boldsymbol{\theta}_1,\boldsymbol{\epsilon}) = \prod_{i=1}^n \int\{ f(\boldsymbol{y}_{i}|\boldsymbol{c}_{i}=0,\boldsymbol{b}_{1i},\boldsymbol{\beta},\sigma) f(\boldsymbol{c}_{i}|\boldsymbol{b}_{3i},\boldsymbol{\eta}) f(\boldsymbol{z}_{i}|\boldsymbol{b}_{2i},\boldsymbol{\alpha}) f(s_i,\delta_i|h_0,\boldsymbol{\gamma_0},\boldsymbol{\gamma_1}) f(\boldsymbol{b}_i|\Sigma)\} d \boldsymbol{b}_i.
    \end{split}
\end{equation}



\subsection{The Proposed Parametric Joint Model}

The proposed framework replace the semi-parametric Cox proportional hazards model by parametric models, e.g.,  Weibull regression models and Log-logistic accelerated failure time models. The AFT models build the relationship between the log of time-to-event and covariates, which is more straightforward and easy to interpret compared to PH models. As the proportional hazards assumption is relatively strong in many cases, the AFT models have less restricted assumptions as well. A comprehensive discussion is provided in \cite{wang2006estimation}.

\subsubsection*{\textit{Parametric Survival Models}}

The Weibull regression models have two alternative representations, a proportional hazards (PH) form and an accelerated failure time (AFT) form. For a clear comparison between the parametric models, we use the AFT forms. The Weibull regression model or the log-logistic AFT model can then be defined as a log-linear model:
\begin{equation}
    log(T_i) = \boldsymbol{x}_{si}^T \boldsymbol{\alpha}_0 + \boldsymbol{b}_i^T  \boldsymbol{\alpha}_1 +\sigma_{\circ} \varepsilon_i,
\end{equation}
where $T_i$ is the survival time, $\sigma_{\circ}$ is a scale parameter, $\boldsymbol{x}_{si}\in \mathbb{R}^{p_4}$ is a vector of time-independent variables for individual $i$, $\boldsymbol{\alpha}_0$ and $\boldsymbol{\alpha}_1$ are fixed effects, and $\varepsilon_i$ is a random variable. The random variable $\varepsilon_i = \frac{log(T_i) - \boldsymbol{x}_{si}^T \boldsymbol{\alpha}_0 - \boldsymbol{b}_i^T  \boldsymbol{\alpha}_1}{\sigma_{\circ} }$ has a density $f_0(\varepsilon_i)$, and a corresponding survival $S_0(\varepsilon_i)$.
Following Lawless in 2003 \cite{lawless2011statistical}, the likelihood function for individual $i$ is given by
\begin{equation}
    \begin{split}
        L_{si} (\boldsymbol{\alpha}_0,\boldsymbol{\alpha}_1,\sigma_{\circ}) &= \Big[ \frac{1}{\sigma_{\circ}} f_0(\varepsilon_i) \Big]^{\delta_i} S_0(\varepsilon_i)^{1-\delta_i}, \\
        &=  \Big[ \frac{1}{\sigma_{\circ}} f_0 \big( \frac{log(T_i) - \boldsymbol{x}_{si}^T \boldsymbol{\alpha}_0 - \boldsymbol{b}_i^T  \boldsymbol{\alpha}_1}{\sigma_{\circ} } \big) \Big]^{\delta_i} 
        S_0 \big( \frac{log(T_i) - \boldsymbol{x}_{si}^T \boldsymbol{\alpha}_0 - \boldsymbol{b}_i^T  \boldsymbol{\alpha}_1}{\sigma_{\circ} } \big)^{1-\delta_i},
    \end{split}
\end{equation}
and the log-likelihood is 
\begin{equation}
    \begin{split}
         \ell_{si} (\boldsymbol{\alpha}_0,\boldsymbol{\alpha}_1,\sigma_{\circ}) &= -\delta_i log(\sigma_{\circ}) + \delta_i log f_0(\varepsilon_i) + (1-\delta_i) S_0(\varepsilon_i) \\ 
         &= -\delta_i log(\sigma_{\circ}) + \delta_i log f_0(\frac{log(T_i) - \boldsymbol{x}_{si}^T \boldsymbol{\alpha}_0 - \boldsymbol{b}_i^T  \boldsymbol{\alpha}_1}{\sigma_{\circ} }) + (1-\delta_i) S_0(\frac{log(T_i) - \boldsymbol{x}_{si}^T \boldsymbol{\alpha}_0 - \boldsymbol{b}_i^T  \boldsymbol{\alpha}_1}{\sigma_{\circ} }).
    \end{split}
    \label{eq:logSurv}
\end{equation}
The Weibull and Log-logistic AFT model assumes the survival data follows a Weibull and log-logistic distributions respectively, and thus the log of survival time follows extreme-value and logistic distributions respectively. The survival for Weibull AFT model is $S_0(\varepsilon)=exp(-e^{\varepsilon})$, and for log-logistic AFT model is $S_0(\varepsilon)=(1+e^{\varepsilon})^{-1}$, and $f_0(\varepsilon)$ is the corresponding density function.

\subsubsection*{\textit{Joint Likelihood with Weibull Regression Model and Log-logistic AFT Models}}

Let the fixed parameters be $\boldsymbol{\theta}_2 = (\boldsymbol{\beta}^T,\boldsymbol{\alpha}^T,\boldsymbol{\alpha}_0^T, \boldsymbol{\alpha}_1^T, \boldsymbol{\eta}^T, \sigma_{\circ})^T$, and the vector of dispersion parameters is $\boldsymbol{\epsilon}=(\sigma, vec({\Sigma}))^T$ as define above in Sec \ref{sec:2.1}. 
The likelihood function of the joint model is
\begin{equation}
    \begin{split}
        L_{h}(\boldsymbol{\theta}_2,\boldsymbol{\epsilon}) = \prod_{i=1}^n \int\{ f(\boldsymbol{y}_{i}|\boldsymbol{c}_{i}=0,\boldsymbol{b}_{1i},\boldsymbol{\beta},\sigma) f(\boldsymbol{c}_{i}|\boldsymbol{b}_{3i},\boldsymbol{\eta}) f(\boldsymbol{z}_{i}|\boldsymbol{b}_{2i},\boldsymbol{\alpha}) f(s_i,\delta_i|\boldsymbol{\alpha}_0,\boldsymbol{\alpha}_1,\sigma_{\circ}) f(\boldsymbol{b}_i|\Sigma)\} d \boldsymbol{b}_i,
    \end{split}
\end{equation}



\subsection{Approximation for Joint Likelihood Inference}

The common way to solve the joint likelihood is the Monte Carlo EM (ECM) algorithms; however, when the vector of parameters has a very high dimension, it converges to the approximate estimates very slowly, and can fail to converge in many cases. A more feasible way is the numerical integration methods such as Gauss-Hermite quadrature method, but it can be computationally consuming in high dimensions. An alternative is the h-likelihood method, which is a numerical integration method as well, but can be much more computationally efficient with reasonable accuracy \cite{ha2003joint,molas2013joint}. The h-likelihood method thus is used to approximate the joint likelihood. By using the h-likelihood method, the integration can be avoided by treating the random effects as parameters in the log-likelihood function. The log h-likelihood function of individual $i$ can be written as the sum of logarithmic terms which contain the term of random effects:
\begin{equation}
\begin{split}
    \ell_{hi} (\boldsymbol{\theta}_2, \boldsymbol{\epsilon}, \boldsymbol{b}_{i}) =&\ log (f(\boldsymbol{y}_{i}|\boldsymbol{c}_{i}=0,\boldsymbol{b}_{1i},\boldsymbol{\beta},\sigma) )
    +\sum_{j=1}^{n_i} \Big[ (1-c_i) logP(c_{ij}=0|\boldsymbol{b}_{3i},\boldsymbol{\eta}) + c_i logP(c_{ij}=1|\boldsymbol{b}_{3i},\boldsymbol{\eta})\Big] +  \\
    &\sum_{j=1}^{n_i} \Big[ (1-z_i)P(z_{ij}=0|\boldsymbol{b}_{2i},\boldsymbol{\alpha}) + z_iP(z_{ij}=1|\boldsymbol{b}_{2i},\boldsymbol{\alpha}) \Big] 
    + log( f(\boldsymbol{b}_i))  + \ell_{si} (\boldsymbol{\alpha}_0,\boldsymbol{\alpha}_1,\sigma_{\circ}),
\end{split}
\end{equation}
where $\ell_{si} (\boldsymbol{\alpha}_0,\boldsymbol{\alpha}_1,\sigma_{\circ})$ is the log-likelihood of AFT models in Eq.(\ref{eq:logSurv}).


The algorithm solves the likelihood function by updating the random effects, fixed effects and dispersion parameters sequentially in each iteration. The choice of initial values of parameters is important as the joint likelihood can be non-identifiable and the optimal estimates are not identical, and the model can hardly be strictly convex and it is easy to yield local optima and fail to converge to the global optimization. It is reasonable to find the initial values by modeling the multiple processes in separate models, and the estimates can then be used as a good start of joint inference. Since the standard errors of parameter estimates are underestimated using the h-likelihood method, an adaptive Gauss-Hermite (aGH) method is proposed to re-estimate the standard errors.
 
\section{Data Application}
\label{sec:applc}

\subsection{Data Description and Model Specification}

We apply the joint models on the dataset \textit{VAX004} provided in the R package \texttt{HHJMs}.
The dataset contains the longitudinal immunization biomarkers and the time-to-infection data of subjects who get HIV vaccines periodically during $36$ months. The study contains $n=194$ subjects in total. \textit{NAb} represents the titer of neutralizing antibodies of the HIV protein and \textit{MNGNE8} represents the averaged level of binding  antibodies. They are the two major longitudinal processes in the data. Let $Y$ denote the original \textit{NAb} which is continuous, and let $Y_{ij}$ denote the original value of \textit{NAb} for individual $i$ at time $j$, $i=1,\dots,n$ and $j=1,\dots,n_i$. Let $Z$ denote the dichotomized \textit{MNGNE8} with $Z_{ij}=1$ representing samples greater than the sample median, and $Z_{ij}=0$ representing the other. The continuous variable $Y$ is left-truncated by a lower limit of quantification (LLOQ) of $d=1.477$. Let $C$ denote the truncation indicator of $Y$, with $C_{ij}=1$ corresponding to the samples smaller than the LLOQ. Observed time to infection of HIV is also contained in the dataset. More details about the data can be found in \cite{yu2018joint}.

The subjects visit the clinic for vaccine injection at $\text{month}=0,1,6,12,18,24,30,36$, and for measurement of immunization biomarkers at the same time points as injection visits and also after half a month of each injection visit. Denote the time points of the measurements as $t_{ij}$ for individual $i$ at time point $j$, the time from the most recent vaccine injection to the most current measurement time as $t_{d_{ij}}$, and the time between two consecutive vaccine injection as $\Delta_{ij}$. Rescale the measurement time $t_{ij}$ from month to year to be $t_{ij}^{\ast}=t_{ij}/12$, and rescale $t_{d_{ij}}$ to be weeks by $t_{d_{ij}}^{\ast}=t_{d_{ij}}*30/7$ by roughly assuming there are 30 days per month.
The covariates are transformed following the study \cite{yu2018joint}. As subjects get vaccine injection periodically, there is a periodic trend in the longitudinal data. A $sin(\cdot)$ transformation is made on the measurement time to capture the periodic trend in data. 
%
%Another covariate is the risk score. There are three categories in the covariate, one of which is used as the baseline and the other two ($\text{risk1}$ and $\text{risk2}$) are treated as dummy variables in the models.

The three major variables are highly correlated, so the joint model is needed to capture the correlation among them. As they are of almost equal importance, separate models are needed to measure the relationship between them and other covariates, rather than measure them in one model with one variable as the response and the others as covariates. Therefore, the HHJMs and proposed parametric joint models are suitable to model the data of interest.
Following the design in \cite{yu2018joint}, define the separate longitudinal models as below. 
\begin{equation}
    \begin{split}
        Y_{ij}|(C_{ij} = 0) &=  \beta_0 + \beta_1 t_{ij}^{\ast} + \beta_2  t_{ij}^{\ast}^2 + \beta_3 sin(\frac{\pi t_{d_{ij}}}{\Delta_{ij}}) + d_1 b_{1i} + \epsilon_{ij}, \\
        %  + \beta_4 \text{risk1}_i + \beta_5 \text{risk2}_i\\
        logit(P(C_{ij} = 1)) &=  \eta_0 + \eta_1 t_{ij}^{\ast} + \eta_2  t_{ij}^{\ast}^2 + \eta_3 sin(\frac{\pi t_{d_{ij}}}{\Delta_{ij}}) + \eta_4 b_{1i},  \\
        %+ \eta_4 \text{risk1}_i + \eta_5 \text{risk2}_i + \\
        logit(P(Z_{ij} = 1)) &= \alpha_0 + \alpha_1 t_{ij} + \alpha_2 sin(\frac{\pi t_{d_{ij}}}{\Delta_{ij}}) + \alpha_3 t_{d_{ij}}^{\ast} + \alpha_4 b_{1i} + d_2 b_{2i} t_{ij},
    \end{split}
    \label{eq:y}
\end{equation}
The truncation indicator $C_{ij}$ is modeled in a generalized linear mixed effects models sharing the same linear predictor of model of $Y_{ij}$. Random effects are introduced into the intercepts.

Three different survival models are considered. The Cox PH model is defined by
\begin{equation}
    \begin{split}
        log\Big[ \frac{h_(T_i|x_i,\boldsymbol{b}_i)}{h_0(t)}\Big] = \gamma_0 x_i + \gamma_1 b_{1i} + \gamma_2 b_{2i}.
    \end{split}
\end{equation}
The Weibull regression model or Log-logistic AFT model is
\begin{equation}
    log(T_i|x_i,\boldsymbol{b}_i) = a_0 x_i + a_1 b_{1i} + a_2 b_{2i}.
\end{equation}

%Let fixed effects $\boldsymbol{\theta}=(\boldsymbol{\beta}^T, \boldsymbol{\eta}^T, \boldsymbol{\alpha}^T, \boldsymbol{\lambda}^T)^T$, random effects $\boldsymbol{b}_i = ({b}_{1i}, {b}_{2i})^T$, and dispersion parameters $\boldsymbol{\epsilon}=(\sigma,vec(\Sigma))^T$, where $\boldsymbol{b}_i\sim N(\boldsymbol{0},\Sigma)$.
%The joint likelihood of the joint model then is ...



\subsection{Approximate Results of the Joint Inference}

%Both h-likelihood method and a more tedious alternative, Gauss-Hermite quadrature method, are used to find the approximate solution of the joint likelihood inference.
%The log h-likelihood function of Eq.(\ref{eq:applc}) of subject $i$ is given by

The results are given in Table \ref{tbl:applc}.

\begin{table}[ht]
\centering
\caption{Estimates of fixed effects in all models using different survival models for \textit{VAX004}}
\begin{tabular}{c|rrr|rrr|rrr}
  \hline
  \multirow{2}{*}{} &  \multicolumn{3}{c|}{Cox} &  \multicolumn{3}{c|}{Weibull (h-likelihood)}   &  \multicolumn{3}{c}{Log-logistic AFT (h-likelihood)} \\
   \hline
   & Estimate & Std.Error & Pvalue & Estimate & Std.Error & Pvalue & Estimate & Std.Error & Pvalue \\ 
  \hline
  \multicolumn{10}{c}{Fixed Effects} \\
  \hline
  $\beta_0$ & 2.079 & 0.062 & 0.00 & 2.079 & 0.063 & 0.00 & 2.012 & 0.040 & 0.00 \\ 
  $\beta_1$ & 0.941 & 0.051 & 0.00 & 0.942 & 0.051 & 0.00 & 0.967 & 0.063 & 0.00 \\ 
  $\beta_2$ & -0.287 & 0.022 & 0.00 & -0.287 & 0.022 & 0.00 & -0.292 & 0.027 & 0.00 \\ 
  $\beta_3$ & 1.488 & 0.028 & 0.00 & 1.488 & 0.028 & 0.00 & 1.539 & 0.034 & 0.00 \\ 
  \hline
  $\eta_0$ & 0.201 & 0.282 & \textcolor{red}{0.48} & 0.198 & 0.287 & \textcolor{red}{0.49} & 0.198 & 0.199 & \textcolor{red}{0.32} \\ 
  $\eta_1$ & -4.285 & 0.478 & 0.00 & -4.268 & 0.478 & 0.00 & -4.216 & 0.504 & 0.00 \\ 
  $\eta_2$ & 1.354 & 0.228 & 0.00 & 1.346 & 0.228 & 0.00 & 1.315 & 0.241 & 0.00 \\ 
  $\eta_3$ & -6.410 & 0.486 & 0.00 & -6.407 & 0.487 & 0.00 & -6.486 & 0.520 & 0.00 \\ 
  \hline
  $\alpha_0$ & -1.700 & 0.127 & 0.00 & -1.700 & 0.128 & 0.00 & -1.676 & 0.121 & 0.00 \\ 
  $\alpha_1$ & 0.165 & 0.016 & 0.00 & 0.164 & 0.019 & 0.00 & 0.149 & 0.008 & 0.00 \\ 
  $\alpha_2$ & 1.938 & 0.140 & 0.00 & 1.940 & 0.140 & 0.00 & 1.981 & 0.144 & 0.00 \\ 
  $\alpha_3$ & -0.046 & 0.007 & 0.00 & -0.046 & 0.007 & 0.00 & -0.048 & 0.007 & 0.00 \\ 
  \hline
  $\gamma_0$ & -0.999 & 0.176 & 0.00 & -0.983 & 0.215 & 0.00 & - & - & - \\ 
  $\gamma_1$ & -1.264 & 0.200 & 0.00 & -1.263 & 0.264 & 0.00 & - & - & - \\ 
  $\gamma_2$ & 2.147 & 0.222 & 0.00 & 2.078 & 0.377 & 0.00 & - & - & - \\
  $a_0$ & - & - & - & 0.063 & - & - & -2.875 & 0.281 & 0.00 \\ 
  $a_1$ & - & - & - & 0.081 & - & - & -1.017 & \textcolor{red}{-} & \textcolor{red}{-} \\ 
  $a_2$ & - & - & - & -0.133 & - & - & 1.232 & \textcolor{red}{-} & \textcolor{red}{-} \\ 
   \hline
   $log(\lambda_{\circ})$ & - & - & - & -103.816 & 13.743 & 0.000  & - & - & -\\
   $\gamma_{\circ}$ & - & - & - & 15.602& 2.067 & 0.000  & -  & - & - \\
   $\sigma_{\circ}$ &  - & - & -  & - & - & - & 0.563 & 0.008 & 0.000 \\
   \hline
   \multicolumn{10}{c}{Dispersion parameters} \\
   \hline
   $\sigma$ & 0.487 & - & - & 0.487 &- &-  & 0.563&- &-  \\
   $r_{12}$ & 0.481 & - & - & 0.484 &- &-  & 0.583 &- &-  \\
   \hline
   \multicolumn{10}{c}{Other secondary parameters} \\
   \hline
   $\eta_4$ & -1.749 & - & - & -1.748 &- &-  & -1.730 &- &-  \\
   $\beta_4$ & 0.448 & - & - & 0.448 &- &-  & 0.460 &- &-  \\
   $\alpha_4$ & 0.398 & - & - & 0.395  &- &-  & 0.285  &- &-  \\
   $\alpha_5$ & 0.162 & - & - & 0.161 &- &-  & 0.169 &- &-  \\
   %$d_1$ &  & - & - & &- &-  & &- &-  \\
   %$d_2$ &  & - & - & &- &-  & &- &-  \\
   \hline
\end{tabular}
\label{tbl:applc}
\end{table}



\section{Simulation}
\label{sec:simul}

The model performance is averaged on $300$ synthetic datasets with individuals $n=100$ of each. We design a similar model as in Sec \ref{sec:applc}. The true parameters are $\boldsymbol{\beta}=(2,1,-0.3,1.5)^T$, $\boldsymbol{\alpha}=(-1.65,0.15,1.8,-0.05,0.4)^T$, $\boldsymbol{\gamma}=(-0.75,-1.5,-2)^T$, the variances of errors in the linear mixed effects model of $Y$ is $\sigma=0.75$, and the covariance between two vectors of random effects as $0.3$ for all individuals. Choose $\text{LLOQ}=2$, let the secondary parameters $d_1=0.5$, $d_2=0.15$. In the Cox PH models, let the baseline hazards models to be $h_0=800$. The simulation of Cox and Weibull regression models are in Table \ref{tbl:siml}.
\begin{table}[ht]
\centering
\caption{Simulation results of Cox and Weibull regression models using h-likelihood}
\begin{tabular}{rr|rrr|rrr}
  \hline
   & & \multicolumn{3}{c|}{Cox} &  \multicolumn{3}{c}{Weibull} \\
  \hline
   & True para & Estimates & Std\_Errors & P-values & Estimates & Std\_Errors & P-values \\ 
  \hline
  $\beta_0$ & 2.000 & 2.144 & 0.096  & 0.000 & 2.153 & 0.096  & 0.000 \\ 
  $\beta_1$ & 1.000 & 0.997 & 0.234 & 0.000 & 0.992 & 0.233 & 0.000 \\ 
  $\beta_2$ & -0.300 & -0.311 & 0.159  & 0.050 & -0.309 & 0.158  & 0.051 \\ 
  $\beta_3$ & 1.500 & 1.395 & 0.081  & 0.000 & 1.389 & 0.081  & 0.000 \\ 
   \hline
  $\alpha_0$ & -1.650 & -1.191 & 0.220  & 0.000 & -1.273 & 0.223  & 0.000 \\ 
  $\alpha_1$ & 0.150 & \textcolor{red}{2.579} & 0.079  & 0.000 & \textcolor{red}{2.751} & 0.076  & 0.000 \\ 
  $\alpha_2$ & 1.800 & 1.522 & 0.376 & 0.000 & 1.554 & 0.382 & 0.000 \\ 
  $\alpha_3$ & -0.050 & -0.071 & 0.038  & \textcolor{red}{0.064} & -0.066 & 0.039  & \textcolor{red}{0.089} \\ 
  \hline
  $\gamma_0$ & -0.750 & -0.644 & 0.132  & 0.001 & -0.463 & 0.139  & 0.001 \\ 
  $\gamma_1$ & -1.500 & \textcolor{red}{-0.614} & 0.134  & 0.015 & \textcolor{red}{-0.310} & 0.240  & \textcolor{red}{0.196} \\ 
  $\gamma_2$ & -2.000 & \textcolor{red}{-1.561} & 0.162 & 0.004 & \textcolor{red}{-1.008} & 0.135  & 0.000 \\ 
   \hline
\end{tabular}
\label{tbl:siml}
\end{table}


The simulation study shows that the parametric Weibull regression models are not more efficient than semi-parametric Cox models as the standard errors of parameters are very similar in both models. There is no obviously higher accuracy in either joint model. Most of the estimates are close to the true parameters; however, three of them ($\alpha_1$, $\gamma_1$, and $\gamma_2$) are not. 

There can be several reasons. Starting values may be far from the optimal which results in an early convergence of the algorithm. As Cox and Weibull are both proportional hazards models, they both have proportional hazards assumption.
Measure the performance of an alternative AFT survival model on the same data, and compare if the alternative model performs better in the three parameters ($\alpha_1$, $\gamma_1$, and $\gamma_2$). If so, the proportional hazards assumption is not valid in the synthetic data.




\section{Conclusions and Discussion}
\label{sec:discuss}

This report extends the joint model for longitudinal data of mixed types with truncated values and survival models by replacing the semi-parametric Cox proportional hazard model by two parametric survival models. It extends the semi-parametric joint modeling framework to parametric framework. The advantage of parametric framework is that the estimates can be more efficient if the distributional assumption of parametric models hold. The performance of parametric models is not guaranteed to be more accurate than semiparametric or nonparametric models. 

The simulation study shows that the parametric Weibull regression model is not necessarily to be more efficient than semi-parametric.

The joint model can be further extended by replacing the mixed effects models into other models, e.g., robust mixed effects models, generalized estimating equations (GEE models), and Bayesian mixed effects models. Other types of survival models are available as well, e.g., 

Compared to proportional hazards models, the accelerated failure time models are more straightforward as they directly measure the relationship between the log of time-to-event without introducing the hazards functions. The models are easier to interpret as well: the change of covariates directly explains the change of the expected log of the survival time. 


Survival models can be extended to frailty models by introducing time-dependent covariates and random effects. %compare using ph or AFT - page6

\section*{Acknowledgement}

The modified R pacakge \texttt{HHJMs.p} and relative R code are provided \href{https://github.com/OliviaJL/parametric-hhjms/blob/main/README.md}{here}. I'd like to thank Dr. Lang Wu for his generous help and precious advice.

\printbibliography

\end{document}
